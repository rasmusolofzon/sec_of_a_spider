\documentclass[a4paper]{article}

\usepackage[utf8]{inputenc}   % Enter your text in ISO-Latin 1
\usepackage{graphicx}
\bibliographystyle{plain}
\usepackage[citestyle=numeric]{biblatex}
\usepackage{textcomp}

\newcommand{\Q}[2]{ \vspace{10pt} \textbf{A-#1} \textit{#2} }
\newcommand{\A}[1]{ #1 }

\title{Home Assignment 3 \\ EITN41}
\author{}
\date{}

\begin{document}
\maketitle

\Q{9} {
  Explain how the zero-knowledge property of a zero-knowledge proof is related to a simulator.
}

\A{
  The idea is that if a verifier could simulate the communication of the proof,
  i. e. fake a transcript of it, and this simulated/faked transcript cannot 
  be distinguished from an actual communication exchange then the verifier
  cannot be said to have learnt anything from the proof. Then the 
  zero-knowledge property is present. The thing is that in an actual
  communication between verifier and prover the probability of a prover
  providing the right choices/information without actually knowing the secret 
  is very, very low 
  (e. g. for 50 iterations of classic examples with graph isomorphism
  or Ali Baba's cave: $2^{-50} = 8.88.. * 10^{-16}$). So, if the prover
  provides correct information in an online setting the verifier
  must conclude that the prover knows the secret, but the verifier still
  have not learned anything about the secret. 
}

\Q{13} {
  Describe two different usages of secret sharing, one where the secret is reconstructed "explicitly", and one where it is not.
}

\A{
  The "explicit" variant is the one first described in the lecture notes.
  
  The "non-explicit" variant is the variant of the second one, used in
  4.3 'Using Homomorphic Encryption'. 
  It is not the actual message $m$ that is decrypted but instead $w^m$ mod q.
}

\Q{14} {
  In the commitment scheme using a hash function (given on the lecture slides),
  is the \textnormal{binding} and
  \textnormal{concealing} properties information theoretic or computational?
}

\A{
  Computational. A hash function is used, and it is inversible. 
  It is only unfeasibly difficult to do so. 
  Most can also be found collisions to.
}

\Q{15} {
  Why is the mix network voting example in the lecture notes divided into 
  registration and voting
  phase? What would happen if the phases were combined and the vote was sent 
  immediately?
}

\A{
  If there is an error (Voter is not registered correctly, a Mix is corrupt etc) then this will 
  not be discovered until the voting already has been carried out. 
  This has the potential of disturbing a re-election triggered by the
  erroneous management, which breaks the fairness property.
}

\Q{19} {
  Consider the blind signature based protocol. No result will be published 
  before all voters has had the
  chance to verify that their vote is indeed correct. How is this important 
  property achieved?
  }

\A{
%  A voter's vote is 'packed in' into one of the first things they send,
%  $h(v,k)$. This is, after going through more functions and the Administrator
%  (who signs it), put on the bulletin board: inside $e_i$ (and also $s_i$, but 
%  that is just $e_i$ signed by voter). 

  A voter can check their vote on the Bulletin Board after the Voting phase,
  then their x is up there. They can see if it is the same value as when they
  calculated it. If it is, they send their secret k so that the Counter
  can extract the voter's vote, v. Before the voter send their k, 
  their vote is unextractable and therefore untalliable.
}

\Q{22} {
  In the slides, two main strategies for making an electronic voting scheme are 
  presented. One is
  that "the vote is posted on the bulletin board in encrypted form, and the 
  person (casting the vote) is not
  anonymous". Describe a scheme like this, and in particular explain how the  
  vote can be counted without sacrificing privacy/anonymity.
}

\A{
  \underline{Threshold encryption scheme}. 
  The vote can be counted because of the homomorphic encryption properties:
  in the encrypted space, the ciphertext can be multiplied and this
  brings the same result as multiplication of the plaintext. 
  We use this in the way that the votes are put in an exponent, this means
  that the votes will be added to each other. The final result is produced
  (in the -1, 1 case) by adding all votes together, positive if there are
  more positive values than negative values and vice versa. Since the 
  votes are encrypted, privacy/anonymity is preserved.
}

\Q{23} {
  In the homomorphic encryption based scheme, why is it important that voters 
  prove that their vote is correct, e.g., either $v_i = -1$ or $v_i = 1$?
}

\A{
  If they do not use -1 or 1, their vote will be weighted more than others.
  The sign of the exponent of $w^{\sum_{t-1}^{m} v_i}$ is what is taken as the
  final result. Two very simple examples to illustrate the difference:
  \{ $w_1 = -1$, $w_2 = 1$, $w_3 = -1$ \}, a correct procedure. Here the result will be negative, as it should be when it is two against one. 
  \{ $w_1 = -1$, $w_2 = 3$, $w_3 = -1$ \}, an incorrect procedure. Here the result
  will be positive, which it should not be when it is
  two against one.
}

\Q{25} {
  Explain why (how) the homomorphic voting scheme in the lecture notes 
  does not have receipt-freeness.
}

\A{
  The Voter is listed, uniquely on the Bulletin Board. This means that they
  (and everyone else) knows that they voted. This does not reveal what they 
  voted for, but at least \textit{that} they voted. So not full 
  receipt-freeness.
}

\end{document}
