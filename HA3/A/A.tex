\documentclass[a4paper]{article}

\usepackage[utf8]{inputenc}   % Enter your text in ISO-Latin 1
\usepackage{graphicx}
\bibliographystyle{plain}
\usepackage[citestyle=numeric]{biblatex}
\usepackage{textcomp}

\newcommand{\Q}[2]{ \vspace{10pt} \textbf{A-#1} \textit{#2} }
\newcommand{\A}[1]{ #1 }

\title{Home Assignment 3 \\ EITN41}
\author{}
\date{}

\begin{document}
\maketitle

\Q{9} {
  Explain how the zero-knowledge property of a zero-knowledge proof is related to a simulator.
}

\A{
  The idea is that if a verifier could simulate the communication of the proof,
  i. e. fake a transcript of it, and this simulated/faked transcript cannot 
  be distinguished from an actual communication exchange then the verifier
  cannot be said to have learnt anything from the proof. Then the 
  zero-knowledge property is present. The thing is that in an actual
  communication between verifier and prover the probability of a prover
  providing the right choices/information without actually knowing the secret 
  is very, very low 
  (e. g. for 50 iterations of classic examples with graph isomorphism
  or Ali Baba's cave: $2^{-50} = 8.88.. * 10^{-16}$). So, if the prover
  provides correct information in an online setting the verifier
  must conclude that the prover knows the secret, but the verifier still
  have not learned anything about the secret. 
}

\Q{13} {
  Describe two different usages of secret sharing, one where the secret is reconstructed "explicitly", and one where it is not.
}

\A{
  TODO
}

\Q{14} {
  In the commitment scheme using a hash function (given on the lecture slides),
  is the \textnormal{binding} and
  \textnormal{concealing} properties information theoretic or computational?
}

\A{
  TODO
}

\Q{15} {
  Why is the mix network voting example in the lecture notes divided into 
  registration and voting
  phase? What would happen if the phases were combined and the vote was sent 
  immediately?
}

\A{
  TENTATIVE: Timing attack? 
}

\Q{19} {
  Consider the blind signature based protocol. No result will be published 
  before all voters has had the
  chance to verify that their vote is indeed correct. How is this important 
  property achieved?
  }

\A{
  TODO
}

\Q{22} {
  In the slides, two main strategies for making an electronic voting scheme are 
  presented. One is
  that "the vote is posted on the bulletin board in encrypted form, and the 
  person (casting the vote) is not
  anonymous". Describe a scheme like this, and in particular explain how the  
  vote can be counted without sacrificing privacy/anonymity.
}

\A{
  TENTATIVE: Blind signature scheme?
}

\Q{23} {
  In the homomorphic encryption based scheme, why is it important that voters 
  prove that their vote is correct, e.g., either $v_i = -1$ or $v_i = 1$?
}

\A{
  TODO
}

\Q{25} {
  Explain why (how) the homomorphic voting scheme in the lecture notes 
  does not have receipt-freeness.
}

\A{
  TODO
}

\end{document}
