\documentclass[a4paper]{article}

\usepackage[utf8]{inputenc}   % Enter your text in ISO-Latin 1
\usepackage{graphicx}

\newcommand{\Q}[2]{ \vspace{10pt} \textbf{A-#1} \textit{#2} }
\newcommand{\A}[1]{ \textbf{Answer:} #1 }

\title{Home Assignment 5 \\ EITN41}
\author{}
\date{}

\begin{document}
\maketitle

\Q{2} {
    In ASN.1, what is the difference between implicit and explicit tagging?
}

\A{
    In implicit tagging the class is context-specific.
    In explicit tagging the class needs an outer tag in order to be
    sufficiently specified.
}

\Q{6} {
    In ASN.1, what is the difference between \texttt{DEFAULT} and 
    \texttt{OPTIONAL}?
}

\A{
}

\Q{13} {
    In ASN.1, an INTEGER has tag value 0x02, which is BER encoded to 0x02.
    A SEQUENCE has tax 0x10, which is BER encoded to 0x30.
    Explain the discrepancy.
}

\A{}

\Q{17} {
    For signed-data in CMS, several signers can sign the same data. 
    How is this feature achieved?
}

\A{
}

\Q{18} {
    Give an example of multiple representation of the same data in CMS.
    Motivate this redundancy.
}

\A{
}

\Q{20} {
    Consider the SignedData type in CMS. 
    The digestAlgorithms are given as a "SET OF DigestAlgorithmIdentifier".
    Since a "SET OF" does not have a particular order, how can we know 
    which digest algorithm corresponds to which signer? Or do we not care?
}

\A{
}

\Q{23} {
    In PKCS \#12, assume that we want to represent a private key. 
    It should be privacy and integrity protected using a password. 
    What is the minimum number of ContentInfo types we have to define
    in order to produce a valid PFX? Which are the ContentTypes we should use?
}

\A{
}

\Q{24} {
    With password integrity mode in PKCS \#12, the MAC is computed over 
    encrypted data. Another strategy could be to compute the MAC over 
    the plaintext and then apply encryption (including or excluding
    the MAC). In general, which variant to use is chosen by protocol or 
    algorithm designers. How is it done in SSL?
}

\A{
}

\end{document}
