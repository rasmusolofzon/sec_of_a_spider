\documentclass[a4paper]{article}

\usepackage[utf8]{inputenc}   % Enter your text in ISO-Latin 1
\usepackage{graphicx}
\bibliographystyle{plain}
\usepackage[citestyle=numeric]{biblatex}
\usepackage{textcomp}

\newcommand{\Q}[2]{ \vspace{10pt} \textbf{A-#1} \textit{#2} }
\newcommand{\A}[1]{ \textbf{Answer:} #1 }

\title{Home Assignment 4 \\ EITN41}
\author{}
\date{}

\begin{document}
\maketitle

\Q{2} { %
    What problem in OTR is solved using the Socialist Millionaire Problem?
}

\A{
    In the first scenario (section 1.4 in the lecture notes)
    it is assumed that Alice and Bob knows each other's public keys,
    mitigating the risk for a potential MitM attack.
    The problem solved by SMP is the agreement on keys,
    here Alice and Bob does not need to know each other's public keys
    in advance.

    % Authentication, and mitigating MitM (Man-in-the-Middle) attacks. 
    % The authentication is ultimately done by a low-entropy shared
    % secret, which is "baked in" into, and protected, in the protocol.
    % The mitigation is achieved by utilising the Diffie-Hellman
    % problem (given $g$, $p$, $g^x$ and $g^y$ it is infeasible 
    % to find $g^{xy}$). In order for both parties to verify 
    % each others public key fingerprint, the fingerprint is
    % concatenated with the negotiated Diffie-Hellman secret 
    % and the shared low-entropy secret. This concatenated value 
    % is then hashed:

    % $$H(PK_A \,||\, PK_B \,||\, g^{x_1y_1} \,||\, "shared \; secret")$$

    % Because of the Diffie Hellman problem, the "shared secret" value 
    % cannot be brute forced. An aspiring MitM attacker also has
    % to guess the secret in one try, otherwise the communicating
    % parts will see that the protocol has failed.
}

\Q{4} {
    How is perfect forward secrecy solved in OTR?
}

\A{
    A MAC is calculated on each message, after every message the MAC key used 
    for that message is released and a new one is calculated/agreed upon
    for the next message. This means that the MAC keys are 'fresh' for only
    one message and that any later use of it directly gives
    the attacker away.
}

\Q{7} {
    Why does not an IdP send a response to an authentication request with the 
    HTTP GET method? What alternatives are there?
}

\A{    % \url{https://www.w3schools.com/tags/ref_httpmethods.asp}
    Partly because a GET is an insecure method, all parameters are 
    included in the URL and GETs can be cached and also stored in 
    browser history
    potentially enabling someone to use an earlier authentication
    without doing a present authentication,
    partly because the parameters are included in the URL, 
    and the size of the response is usually larger
    than the Subjects user-agent (e.g. browser)
    can handle.

    An alternative is an artifact binding.
}

\Q{9} {
    Name and describe four problems in SAML. You should be able to provide a 
    more detailed description for one of them.
}

\A{
    TODO
}

\Q{11} {
    Describe and compare "discovery" in SAML and OpenID.
}

\A{
    TODO
}

\Q{12} {
    In a certain sense, there are three different types of communication in 
    SAML and two in OpenID. Describe them and explain the difference.
}

\A{
    TODO
}

\Q{18} { %
    In OAuth, explain why the access token must not be cached, and how this is 
    achieved.
}

\A{
    If the token is cached, 

    The Authorisation Server returns an access token with a 200,
    the data in JSON format. The 200 has header options
    telling the receiver (Client) to not cache the token (and data).

    \texttt{Cache-Control: no-store}, this is the \textit{don't cache}
    command, if you will. 
    \url{https://tools.ietf.org/html/rfc7234#section-5.2.1.5}

    \texttt{Pragma: no-cache}, this is mostly used for backwards 
    compatibility, \url{https://tools.ietf.org/html/rfc7234#section-5.4}.
    If HTTP 1.0 is used, Cache-Control is not understood,
    in this case Pragma specifies \textit{don't cache}. 
    If Cache-Control is understood, Pragma is ignored.

    % Interestingly: \url{https://www.w3.org/Protocols/rfc2616/rfc2616-sec14.html#sec14.32}
}

\Q{20} { %
    What is a grant? Name and describe a few different grants.
}

\A{
    It is a sort of attestation, or authorisation promise. 
    The Resource Owner grants a Client access to certain resources
    the Owner owns. This can be used by the Client to request
    an Access Token from the Authorisation Server.
    The process for this can be different for different types
    of grants. The four types / classes described in the lecture notes are:

    \begin{itemize}
        \item \textbf{Authorisation code - }
            the Client receives a code, which is used
            when they request access from Authorisation Server.
        \item \textbf{Implicit grant - }
            the Client receives a token directly from the Authorisation
            Server.
        \item \textbf{Resource Owner Password Credentials grant - }
            Resource Owner gives Client their own login credentials
            for the Authorisation Server. Using these, the Client 
            receives a token from the server.
        \item \textbf{Client Credentials grant - }
            Here the Resource Owner is not involved. The Client
            directly from Authorisation Server after provding
            their own credentials.
    \end{itemize}

}

\end{document}
