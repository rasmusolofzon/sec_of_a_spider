\documentclass[a4paper]{article}

\usepackage[utf8]{inputenc}   % Enter your text in ISO-Latin 1
\usepackage{graphicx}
\bibliographystyle{plain}
\usepackage[citestyle=numeric]{biblatex}
\usepackage{textcomp}

\newcommand{\Q}[2]{
  \textbf{#1} \textit{#2}
 }
\newcommand{\A}[1]{ #1 }
\newcommand{\Grade}[2]{ 
  \textbf{Grading motivation:} #2 \\ 
  \hspace*{\fill} \textbf{Grade:} #1 
}

\title{Home Assignment 1 \\ EITN41}
\author{}
\date{}

\begin{document}
\maketitle

\Q{A-4} {
  Give two common ways to prove/make probable that the person making
  a card-not-present transaction is in physical possession of the card. Compare
  the two alternatives in terms of security.
}

\A{
  Two efforts to increase security during a card-not-present transaction dis-
  cussed during the course are SET and 3D Secure. These two methods provide
  authentication that the buyer is in fact the one registered to the card used.
  I find it redundant to explain the whole protocols in detail since the question
  in hand is just to compare the security of them which we can do without this.
  3D Secure provides user authentication typically via the use of a password
  the cardholder must enter at the time of purchase. Other authentications can be
  chosen by the bank such as BankID (this author has touchID via BankID app
  for example). 3D Secure can also protect the cardholder against phising sites
  by allowing the display of a Personal Assurance Message set by the cardholder
  on card activation, so this person can verify that the pop-up box in which to
  enter the password is from the bank. I.e two factor authentication. To stop
  attackers from intercepting a card and choose a password or such before the
  user can, the enrollment into the 3D Secure system is fortified by a number of
  options. Either the bank can choose to let you activate the card face-to-face
  where they can check your ID, or via their website where its assumed only you
  can log into your personal pages/account management. Or a third option is
  available where the card is activated on the first purchase made and password is
  chosen at that time. This is protected by the use of personal questions assumed
  only the intended user knows.

  SET, Secure Electronic Transaction, provides more than just two factor au-
  thentication but failed to gain traction on the market due to user friendliness
  issues. In this authentication method the merchant will never get a hold on the
  card details and the bank will never know the items ordered. The use of PKI
  and certificates are implemented, which is the user friendliness issues mentioned.
  This protocol relies on the use of the dual signature technique and encryption
  via the gateways public key to keep the order information from being read by
  the bank and the payment information from being read by the merchant. The
  dual signature consists a hash/digest of two already hashed parts, the payment
  information and the order information. with the usage of hashes and signatures
  this protocol also provides integrity of the data. In summary, 3D secure pro-
  vides us with authentication of the cardholder, an optionally authentication of
  the bank. While SET provides us with two factor authentication, data integrity,
  and confidentiality of information. Since 3D Secure typically uses a password
  the strength of this protocol lies on the cardholder. If a bad/easy password is
  chosen this is susceptible to guessing attacks, for example if the password is the
  name of the pet. Or if the personal questions are easy to guess. A possible
  security 
  aw of the SET method is the breach of the PKI or loss of private key.
  PS sorry for this wall of text.
}

\Grade{0.0}{}

\Q{A-6} {In SET, why is the Payment Information first symmetrically encrypted
and not immediately encrypted with the Gateway's public key?}

\A{
  The payment information, PI, is symmetrically encrypted with DES in SET
  \textbf{due to the speed of this encryption algorithm}. Symmetric encryption is
  a lot faster than asymmetric, so encrypting everything with the gateways public
  key would take a lot more time. The key used by the cardholder is then en-
  crypted with the use of the public key of the gateway and sent together with the
  encrypted PI. When the merchant then relays the information to the payment
  gateway after order verification the gateway can unwrap/decrypt the key used
  to encrypt PI. Then in turn decrypt the PI and verify this with OIMD in the
  dual signature.
}

\Grade{0.0}{}

\Q{A-12} {How is mutual authentication between issuer and cardholder achieved
in VbV (3D Secure)?}

\A{
  As mentioned in question \textbf{A-3} the 3D Secure protocol offers two factor authentication from the bank side via the use of a Personal Assurance Message
  to be displayed with each purchase, verifying the pop-up window was issued by
  the bank. And from the client side via the password entered by the cardholder,
  which is assumed secret to only this person.
}

\Grade{0.0}{}

\Q{A-14} {The multiplicative property of RSA provides for blind signatures. What
is meant by "the multiplicative property of RSA"?}

\A{
  This simply means that $E(m_1) * E(m_2) = E(m_1 * m_2)$. Where E(x) is the
  encryption of x and * is an arbitrary operation.
}

\Grade{0.0}{}

\Q{A-15}{When requesting a blind signature, why must Alice keep r secret?}

\A{...}

\Grade{0.0}{}

\Q{A-18} {When Alice buys something from Bob using the untraceable E-cash
scheme, why is it impossible for Bob to learn the identity of Alice?}

\A{
  ...
}

\Grade{0.0}{}

\Q{A-28} {Compare the PayWord protocol and the Peppercoin-like protocol in the
lecture notes from the point of view of the customers, both in terms of what they
pay, and in terms of what they need to compute to make a purchase.}

\A{
  ...
}

\Grade{0.0}{}

\Q{A-29} {What  is  meant  by  a  probabilistic  payment?   How  does  the  
Electronic  Lottery  Tickets  scheme  differ from Peppercoin from the user's perspective?
How do they differ from the Merchant's perspective?}

\A{
  ...
}

\Grade{0.6}{
  Important distinction: In the example with Alice making 100 purchases, she will \textit{on average} "be debited an amount of 100 SEK once and 99 times
  nothing". It is probable, but not guaranteed. In a larger timescale it will average out.

  An important factor this answer does not treat is where the psychological load is placed: 
  In (ELT), the psychological load is placed on the User in that they sometimes pay more than they have
  actually spent. In Peppercoin the User never pays more than they have actu-
  ally spent and the psychological load is taken by the Bank.

  Otherwise the information seems correct according to the lecture slides.
}

\end{document}