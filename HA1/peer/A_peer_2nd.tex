\documentclass[a4paper]{article}

\usepackage[utf8]{inputenc}   % Enter your text in ISO-Latin 1
\usepackage{graphicx}
\bibliographystyle{plain}
\usepackage[citestyle=numeric]{biblatex}
\usepackage{textcomp}

\newcommand{\Q}[2]{
  \vspace{10pt} \textbf{#1} \textit{#2}
 }
\newcommand{\A}[1]{ #1 }
\newcommand{\Grade}[2]{ 
  \textbf{Grading motivation:} #2 \\ 
  \hspace*{\fill} \textbf{Grade:} #1 
}

\title{Home Assignment 1 \\ EITN41}
\author{}
\date{}

\begin{document}
\maketitle

Grading: $(0.9+0.8+0.7+1.0+1.0+0.8+1.0)=$ \textbf{6.2}

\Q{A-7} {Is SSL required in SET? Motivate your answer.}

\A{
  No. SET (Secure Electronic Transaction) aimed to provide application level
  confidentiality and integrity protection of the transmitted information. It used PKI infrastructure,
  identities based on X.509 certificates, RSA signatures, symmetric encryption and special
  software to be installed. It is hard to believe that SSL would be required (it was also very early
  for SSL at the time of drafting SET) as conceptually the protocol itself is built on many of the
  same exact components and theory. SET was promised to deliver confidentiality, authentication,
  integrity and even non-repudiation by itself.
}

\Grade{0.9}{
  A correct answer. It seems the details were taken from the Wikipedia page, \url{https://en.wikipedia.org/wiki/Secure_Electronic_Transaction} . 
  The information there supports what is described in the answer.
  Unfortunately the references for e. g. mentioning of specific technologies 
  (X.509, RSA etc.) on that Wikipedia page are down.
  A comparison between SET and SSL can be seen here, \url{http://www.bios-school.org/essay-sample-on-differences-between-set-and-ssl.html} ,
  where the notion that the role fulfilled by SSL is also filled by SET
  is augmented.
}

\Q{A-11} {The acronyms ACS and ADS are both related to VbV (3D Secure). Explain
them briefly.}

\A{
  ACS (Access Control Server) is a authentication (often just a password validation with
  the issuing bank) service provided in the third domain, the so called Interoperability Domain in
  the 3D Secure scheme.This is often outsourced, meaning that access control is performed on a
  third-party web server/address. ADS (Activation During Shopping) is a scheme to offer/force
  non-participating users to “sign up” while in the process of using their card online. Some argue
  that ADS poses more risk than advantages since a) the shopping-minded card user is likely to
  “just get it done” and chose e.g. weak passwords (later used by ACS), or b) this just like ACS
  often resides at a third-party, introducing doubt or even phishing vulnerabilities into the concept.
}

\Grade{0.8}{
  ACS in Interoperability domain not correct according to page 15-16 of
  \url{https://usa.visa.com/dam/VCOM/download/merchants/verified-by-visa-acquirer-merchant-implementation-guide.pdf}. 
  It is in the Issuer domain. This is, however, 
  not clearly stated in the lecture slides (page 9) but still expressed there 
  ("\textit{On the issuer's side the most important part is the Access Control Server 
  $[...]$}")
  Correct about outsourcing/third-party mention (lecture slides).
  The insight about a hurried customer choosing weak passwords 
  (and also phishing apprehensions) is also
  mentioned in 
  \url{http://sec.cs.ucl.ac.uk/users/smurdoch/papers/fc10vbvsecurecode.pdf}
  , page 338.
}

\Q{A-12} {How is mutual authentication between issuer and cardholder achieved in
VbV (3D Secure)?}

\A{
  This is in part achieved by the ACS (Access Control Server) described in question A-11
  from the perspective of authentication of the cardholder by the issuer. In practice, a so called
  Payer Authentication Request (PAReq) is sent to the user which in essence opens an URL to
  an authentication website.

  The other way around, the authentication of the issuer by the cardholder is more problematic as
  third-parties often are hosting the ACS/website on other domains than the bank’s. This is prone
  to phishing attacks. If the browser would visit the bank’s site the cardholder could rely on its
  certificate. In most practical cases, the cardholder simply has to trust a few logos and no true
  authentication is performed.
}

\Grade{0.7}{
  Good point about risk for phishing attacks. However,
  the Issuer provides the User's personal message
  as a kind of authentication.

  The User's authentication towards the Issuer is not explicitly mentioned here,
  but by the reference to question A-11 and the parens 
  \textit{"(often just a password validation with the issuing bank)"}
  the question could be deemed answered.
}

\Q{A-14} {The multiplicative property of RSA provides for blind signatures. What is meant by ”the multiplicative property of RSA”?}

\A{
  Mathematically speaking it means that the product of the encrypted messages is equal
  to the encrypted product of the plaintext messages, E(m1) * E(m2) = E(m1 * m2).
}

\Grade{1.0}{
  Correct, according to \url{http://www.eit.lth.se/fileadmin/eit/courses/eit060/lect/Lect2-3.pdf}, slide 23 and slide 6 of 'ElectronicPayments\_2' in this course. 
}

\Q{A-22} {Briefly explain the differences between session-level aggregation,
aggregation by intermediation and universal aggregation.}

\A{
  In the context of micropayments, these techniques are used to group together small
  payments into larger ones, thus reducing the fees.
  \begin{itemize}
    \item{Session-level: All small purchases made during e.g. a user’s shopping session are
    aggregated until the end of the session, then processed together. An example could be
    a user that buys/downloads many songs that only costs very little individually, but during
    a whole session (e.g. a day) the aggregated amount for all is significant and worth the
    processing fees for the merchant.}
   \item{Universal: Aggregation of microtransaction between multiple users and multiple
    merchants into macro transactions, not just one-user-to-one-merchant as above.
    Probabilistic payment schemes are often used (see question A-29 below).}
   \item{Intermediation: An intermediary (e.g. third-party payment provider) gathers all
    microtransactions from multiple users and/or merchants and only processes them when
    it has reached a certain threshold. Example could be a virtual wallet which has virtual
    coins purchased once, then used in small increments on multiple merchants or
    occasions.}
  \end{itemize}
}

\Grade{1.0}{
  All correct, according to lecture notes.
}

\Q{A-29} {What  is  meant  by  a  probabilistic  payment?   How  does  the  
Electronic  Lottery  Tickets  scheme  differ from Peppercoin from the user's perspective?
How do they differ from the Merchant's perspective?}

\A{
  The concept of probabilistic payments is often used as a technique to accomplish
  universal aggregation (see question A-22 above) in the context of microtransactions where rare
  larger payments are “cheaper” to do due to the fee structure than many small ones. The idea is
  to instead of making a micropayment of e.g. \$x, the user makes a macro payment of \$y with the
  probability x/y (and pays nothing with the probability 1-x/y). Example; Let x=1, y=100. Then,
  user pays \$100 with probability 0.01 (1/100) and \$0 with probability of 0.99 (1-0.01). On average
  though, he pays exactly \$1.
  The concept is denoted electronic lottery tickets and though it of course evens out in the long
  run, a user can be hit with (several) large payments immediately, making the system feel unfair.
  A user who only buys very few things, can in theory also be overcharged. An improvement,
  Peppercoin, aims to guarantee that a user never pays more than is “spent”. The problem is
  moved over to the bank’s side of the transaction instead. For the merchant, the two solutions
  are equal. He gets paid the large amount when it is time for it, probabilistically speaking.
}

\Grade{0.8}{
  Correct answer, according to lecture notes and slides.

  Some things: "the problem is moved over to the bank's side
  of the transaction instead" could be clarified; the bank pays the 
  difference. There are also some differences between ELT and Peppercoin in 
  what has to be provided by different parties, e. g. in Peppercoin
  Merchant has to provide a signature and a mapping function on the information
  from Alice, in ELT they (the Merchant) have to calculate a hash chain for 
  Alice.
}

\Q{A-34} {How  is  the  difficulty  in  Bitcoin  block  hashing  adapted  so  that  it  
(almost)  always  takes  about  10 minutes for the system to produce a new block, 
regardless of the computational power that enters the system?}

\A{
  Bitcoin uses a "proof of work" which regulates the computation efforts of new blocks.
  Specifically, a hash must be computed with a constrained result (must be lower than a threshold
  number). It is this number, which is re-evaluated every 14 days, that is determined to aim for a
  10 minute block creation time. In other words, every two weeks, if much more computational
  power has entered the network, the threshold is adjusted so that the computation gets harder
  and the time is held at a constant 10 minutes.
}

\Grade{1.0}{
  Correct according to lecture notes.
}

\end{document}
