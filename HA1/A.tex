\documentclass[a4paper]{article}

\usepackage[utf8]{inputenc}   % Enter your text in ISO-Latin 1
\usepackage{graphicx}
\bibliographystyle{plain}
\usepackage[citestyle=numeric]{biblatex}

\newcommand{\Q}[2]{
  \textbf{#1} \textit{#2}
 }
\newcommand{\A}[1]{ #1 }

\title{Home Assignment 1 \\ EITN41}
\author{Rasmus Olofzon, 9104192472}
\date{}

\begin{document}
\maketitle

\Q{A-4} {What  is  the  difference  between  a  three-party  scheme  and  a  four-party  
scheme  for  credit  card  payments?}

\A{Blah blah blah.}

\Q{A-7} {Is SSL required in SET? Motivate your answer.}

\Q{A-13} {What is the difference between authorization and authentication in VbV 
(3D Secure)?}

\Q{A-23} {In  PayWord,  a  unit  could  be,  e.g.,  one  cent  (or  one  {\"o}re),  
so  even  though  the  payments  are  "micro", the hash chains could be pretty long.  
Could this pose a storage problem to Alice, who has to generate the entire chain when 
(before) she makes her  first purchase from a merchant?}

\Q{A-27} {In the PayWord protocol, give the Bank's algorithm for verifying how much money 
should be taken from the user's account.}

\Q{A-29} {What  is  meant  by  a  probabilistic  payment?   How  does  the  
Electronic  Lottery  Tickets  scheme  differ from Peppercoin from the user's perspective?
How do they differ from the Merchant's perspective?}

\Q{A-31} {How much is the transaction fee for a Bitcoin transaction and how is it 
determined?}

\Q{A-34} {How  is  the  difficulty  in  Bitcoin  block  hashing  adapted  so  that  it  
(almost)  always  takes  about  10 minutes for the system to produce a new block, 
regardless of the computational power that enters the system?}

% \begin{thebibliography}{9}
  % \bibitem{cossim} \url{https://en.wikipedia.org/wiki/Cosine_similarity}
% \end{thebibliography}

\end{document}