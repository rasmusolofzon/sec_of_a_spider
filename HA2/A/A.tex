\documentclass[a4paper]{article}

\usepackage[utf8]{inputenc}   % Enter your text in ISO-Latin 1
\usepackage{graphicx}
\bibliographystyle{plain}
\usepackage[citestyle=numeric]{biblatex}
\usepackage{textcomp}

\newcommand{\Q}[2]{ \vspace{10pt} \textbf{#1} \textit{#2} }
\newcommand{\A}[1]{ #1 }

\title{Home Assignment 2 \\ EITN41}
\author{Anon}
\date{}

\begin{document}
\maketitle

\Q{A-6} { %j
  What is the purpose of the random values $R_1$ in a Mix?
}

\A{
  The purpose of using random values $R_1$ in a Mix is
  to hide the correspondence between inputs and outputs to the Mix.

  Without $R_1$ the input to the Mix would be
  $K_1(K_a(R_0,M),A)$. It would be trivial for an attacker to encrypt
  the output $K_a(R_0,M),A$ with the Mix's public key $K_1$
  and do a lookup in a list of recorded input messages 
  and find the corresponding input, effectively linking
  origin and destination.
}

\Q{A-7} { %j
  When sending a mail through several Mixes, there are several public keys 
  involved: $K_1,K_2, . . . ,K_n$ and $K_a$. 
  What happens if one does not use $K_a$? 
  Does this risk the anonymity of the sender?
}

\A{
  If $K_a$ is used, the output of the mixes look like this:
  \begin{displaymath}
    K_a(R_0,M),A
  \end{displaymath}

  Without using $K_a$, the output should be:
  \begin{displaymath}
    R_0,M,A
  \end{displaymath}

  This means that the randomness $R_0$ is superfluous 
  and the message $M$ is sent in plaintext. This does not
  directly mean that the sender's anonymity is risked, but 
  it could be deduced from $M$.
}

\Q{A-8} {
 Briefly explain how using several Mixes versus an onion routing 
 circuit differ both in terms of latency and in cryptographic 
 primitives used for encrypting the traffic. 
}

\A{
  blah
}

\Q{A-12} {
  Regarding replay attacks on Mixes, two protections are suggested in the lecture notes. 
  Which? Would you say that any of them is the better choice?
  Show how the two strategies can be combined and how this
  can make the protection more efficient.
}

\A{
  \begin{enumerate}
    \item Calculate hash for each input message and store this.
    Do a lookup of all subsequent messages, if one matches, 
    throw away.
    \item Include a timestamp in each input, to verify that 
    the message is 'fresh'.
  \end{enumerate}

  One difference to consider is the storage and computational
  issues with using hashes: computational for calculating
  hashes and doing later lookups, storage for storing hashes of
  all messages, whose cardinality can be quite large.

  TODO: write abt combination
}

\Q{A-13} { %j
  It is straightforward to generalize the N - 1 attack to an 
  N - k attack, 0 $<$ k $<$ N. Describe the N - k attack.
}

\A{
  The general idea of a N - 1 attack is to reduce the size of
  the anonymity set for a user. An attacker controls exactly N - 1 
  inputs to the Mix (total inputs = N). The attacker knows
  the recipients to their N - 1 messages and therefore the only 
  message they didn't send is the one sent by the victim.
  The generalization of this is that an attacker controls N - k
  inputs. Even if this attack is not as effective as the N - 1
  attack, it still reduces the anonymity set of a target user 
  to the size k, which from the attackers point of view
  is something positive.  
}

\Q{A-15} {
  In the disclosure attack on mixes, 
  explain m, N, n and why a Mix is insecure if $m \leq [N / n]$.
}

\A{
  \begin{itemize}
    \item \textbf{m} The result of a successful attack. 
    \textit{m} sets with exactly one recipient, then these \textit{m}
    recipients are \underline{Alice's communication partners}. 
    \item \textbf{N} The total number of users in the anonymity system.
    \item \textbf{n} The number of receivers in a batch.
  \end{itemize}
}

\Q{A-26} {
  A TCP handshake consists of the client and the server exchanging 
  three messages: SYN, SYN-ACK and ACK. 
  Explain why, in Tor, Alice can connect to a webserver and expect 
  the TCP handshake with the
  server to be performed with low latency.
}

\A{
  blah
}

\Q{A-28} {
  Show that the SSL/TLS handshake, when RSA is used, 
  does not provide perfect forward secrecy.
}

\A{
  blah
}

% \begin{thebibliography}{9}
  % \bibitem{cossim} \url{https://en.wikipedia.org/wiki/Cosine_similarity}
% \end{thebibliography}

% Uploaded file: A.pdf
% File saved as: a4c2b8ad79eabee2_HA1_solution.pdf

\end{document}
